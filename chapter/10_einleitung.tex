\section{Einleitung}
% (≈ 1,5 Seiten)

\subsection{Motivation}
% Warum Empfehlersysteme wichtig für digitalen Journalismus sind
Die digitale Transformation hat die Art und Weise, wie Nachrichten konsumiert werden, grundlegend verändert. 
Mit der exponentiell steigenden Informationsdichte im Internet wird es für Nutzer zunehmend schwieriger, 
relevante Nachrichteninhalte zu erkennen.
In diesem Kontext spielen Empfehlungssysteme eine entscheidende Rolle, indem sie nicht nur dabei helfen, 
die Informationsüberlastung zu verringern, sondern auch die Leseerfahrung der Nutzer durch personalisierte Inhalte
zu verbessern (\cite{wu_personalized_2022}). \newline
In der Medienbranche werden \ac{ES} weitverbreitet eingesetzt, um quantitative Metriken wie die Nutzerbindung, Verweildauer, Artikel pro Sitzung zu steigern. Oft werden dabei monetäre Ziele verfolgt, aber auch das Nutzerelebnis kann durch \ac{ES} erheblich verbessert werden.


\subsection{Problemstellung}
% Herausforderungen bei Schwäbisch Media: viele Artikel, Nutzerdiversität, kalter Start

\subsection{Zielsetzung und Aufbau der Arbeit}
% Ziel: Entwicklung & Optimierung auf GCP
% Ankündigung V-Struktur
% Relevanz für Schwäbisch Media