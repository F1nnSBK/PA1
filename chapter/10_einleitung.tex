\section{Einleitung}
% (≈ 1,5 Seiten)

\subsection{Motivation}
% Warum Empfehlersysteme wichtig für digitalen Journalismus sind
Die digitale Transformation hat die Art und Weise, wie Nachrichten konsumiert werden, grundlegend verändert. 
Mit der exponentiell steigenden Informationsmenge im Internet wird es für Nutzer zunehmend schwieriger, 
relevante Nachrichteninhalte zu erkennen, was einen kritischen Bedarf an Filtermechanismen schafft. 
In diesem Kontext spielen Empfehlungssysteme (ES) eine entscheidende Rolle, 
indem sie die Informationsüberlastung von einem Hindernis für das Engagement in eine Chance für personalisierte 
Inhalte verwandeln und so die Leseerfahrung der Nutzer verbessern (Wu et al. 2022).
\newline
Über ihre Funktion als reine Filter hinaus werden sie in dir Medienbranche weitverbreitet eingesetzt, 
um quantitative Metriken wie die Nutzerbindung, Verweildauer oder die Artikel pro Sitzung zu steigern. 
Die empirischen Belege aus verschiedenen Domänen, darunter E-Commerce, Straming-Dienste
und Nachrichtenportale, zeigen einen direkten kausalen Zusammenhang zwischen verbesserten
Nutzerbindungsmetriken und greifbaren finanziellen Vorteilen wie Umsatz und Abonnements. \cite{}


\subsection{Problemstellung}
% Herausforderungen bei Schwäbisch Media: viele Artikel, Nutzerdiversität, kalter Start

\subsection{Zielsetzung und Aufbau der Arbeit}
% Ziel: Entwicklung & Optimierung auf GCP
% Ankündigung V-Struktur
% Relevanz für Schwäbisch Media