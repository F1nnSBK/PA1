\section{Fazit und Ausblick} % Kapitel 6

\subsection{Zusammenfassung zentraler Erkenntnisse}
Die vorliegende Arbeit demonstrierte erfolgreich die Konzeption, 
Implementierung und Optimierung eines hybriden Empfehlungssystems zur 
Steigerung der Nutzerbindung für die \ac{SV-Gruppe}. 
Es wurde der empirische Nachweis erbracht, dass ein datengetriebener 
Ansatz zur Gewichtung der Systemkomponenten zu einer signifikanten Verbesserung 
der Empfehlungsqualität führt.

Die zentrale Erkenntnis ist, dass eine hybride Architektur, 
die ein kontextuelles \ac{CBF}-Modell mit einem personalisierten 
\ac{CF}-Modell kombiniert, den reinen Baseline-Strategien signifikant überlegen ist. 
Die systematische Optimierung mittels Optuna identifizierte eine robuste 
Konfiguration, in der die thematische Relevanz des unmittelbaren Kontexts dominiert, 
aber durch kollaborative Signale entscheidend angereichert wird. 
Die Maximierung der \ac{nDCG}@10-Metrik dient hierbei als validierter Indikator 
für eine verbesserte Nutzererfahrung.

Darüber hinaus wurde gezeigt, dass der Prototyp auf Basis der bestehenden 
\ac{GCP}-Infrastruktur die definierten Anforderungen an Skalierbarkeit und 
Latenz erfüllt. Die Arbeit liefert somit nicht nur ein theoretisch 
fundiertes Modell, sondern auch eine praktisch umsetzbare und leistungsfähige 
technische Blaupause für den produktiven Einsatz.

\subsection{Zukünftige Optimierungsmöglichkeiten}
Aufbauend auf den gewonnenen Erkenntnissen ergeben sich mehrere vielversprechende
Richtungen für die Weiterentwicklung des Systems.

Ein zentraler nächster Schritt ist die Durchführung von Online-A/B-Tests, 
um die in der Offline-Evaluation gemessenen Qualitätsgewinne unter realen 
Bedingungen zu validieren. Hierbei würde das entwickelte \ac{ES} gegen das 
bestehende System ausgespielt, um den kausalen Einfluss auf zentrale 
Geschäftsmetriken wie die „Artikel pro Session”, 
die Klickrate und die Verweildauer zu quantifizieren.

Darüber hinaus könnte die aktuell genutzte, einfache gewichtete 
Hybridisierung durch mehrstufige Architekturen ersetzt werden. 
Ein vielversprechender Ansatz wäre ein regelbasiertes Re-Ranking der 
Top-Empfehlungen. In diesem zweiten Schritt könnten die Kandidaten basierend 
auf weiteren Kriterien umsortiert werden, beispielsweise um Artikel hinter einer 
Paywall für abonnierte Nutzer zu priorisieren, eine höhere thematische 
Diversität sicherzustellen oder die Aktualität der Nachrichten zu berücksichtigen.

Ein weiterer vielversprechender Anwendungsfall liegt in der Entwicklung 
personalisierter Redaktionsassistenten. Die dem \ac{CBF}-Modell zugrundeliegende 
Technologie zur semantischen Ähnlichkeit könnte Redakteuren in Echtzeit passende 
interne Verlinkungen vorschlagen, beim Tagging von Artikeln unterstützen 
oder aufzeigen, welche Inhalte aus dem Archiv zu einem aktuellen Thema relevant 
sind. Dies würde nicht nur die Effizienz der Redaktionsprozesse steigern, 
sondern auch die Qualität und Vernetzung der Inhalte verbessern.