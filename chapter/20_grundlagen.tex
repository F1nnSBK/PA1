\section{Theoretische Grundlagen}
% V-Ansatz Teil 1 – Allgemeiner Rahmen
Das Forschungsfeld der Empfehlungssysteme ist breit und umfasst zahlreiche methodische
Ansätze. In der Praxis haben sich jedoch zwei grundlegende Paradigmen als besonders einflussreich erwiesen:
Das \ac{CBF}, das auf Inhaltsmerkmalen der zu empfehlenden Items basiert und das \ac{CF}, das aus dem kollektiven
Verhalten der Nutzer lernt. Aufgrund der komplementären Stärken und Schwächen dieser beiden Paradigmen
stellen moderne hybride Architekturen, die beide Ansätze kombinieren, den aktuellen
Stand der Technik dar. Diese Arbeit konzentriert sich auf die systematische Optimierung eines solchen hybriden Systems.

\subsection{Empfehlungssysteme im Überblick}
\subsubsection{Content Based Filtering}
Das \ac{CBF} basiert auf der Idee, Artikel zu empfehlen, die thematisch ähnlich zueinander sind. Dafür werden
die Artikeltexte 
\subsubsection{Collaborative Filtering}

\subsubsection{Hybrid Filtering}

% Cold Start, Vielfalt, Skalierbarkeit
\subsubsection{Das Cold Start Problem}

\subsubsection{Serendipität}
\label{sec:serendipitaet}

\subsubsection{Skalierbarkeit}

\subsection{Qualitätsdimensionen und Erfolgsmetriken}
% Precision@k, Recall, NDCG, APS
% Serendipity, Diversität

\subsection{Multikriterielle Systeme und Trade-offs}
% Relevanz vs. Vielfalt
% Pareto-Front (nur Konzepte)

% Fußnote: Optional GPT-Prompt-Zitation bei Bedarf
% \footnote{Einige Abschnitte wurden durch KI-gestützte Schreibunterstützung (ChatGPT) vorformuliert und redaktionell überarbeitet.}
