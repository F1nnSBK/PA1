\section{Theoretische Grundlagen}
% V-Ansatz Teil 1 – Allgemeiner Rahmen
Für moderne \ac{ES} existieren drei fundamentale Ansätze. 
Der \ac{CBF}-Ansatz basiert auf dem Inhalt von Artikeln, um semantisch ähnliche Artikel zu 
identifizieren. Im Gegensatz dazu konzentriert sich der \ac{CF}-Ansatz auf den Nutzer, 
indem er die Lesehistorie eines Nutzers als Grundlage verwendet, 
um personalisierte Empfehlungen zu generieren. Zusätzlich können dem Modell Informationen wie 
Standortdaten oder die Länge eines Artikels hinzugefügt werden, um die Genauigkeit der 
Empfehlungen zu erhöhen. Der dritte Ansatz kombiniert \ac{CBF} und \ac{CF}, 
dabei vereinen hybride Ansätze die Vorteile der beiden zuvor genannten Methoden (\cite{raza_news_2020}).
Eine einfache Implementierung solcher hybrider Methoden kann 
durch die gewichtete Summe der beiden Scores erreicht werden, während fortgeschrittene Methoden, 
wie etwa das Two Tower Modell oder Ensembel-Methoden, anspruchsvollere Ergebnisse liefern.

\subsection{Empfehlungssysteme im Überblick}
% CBF, CF, Hybrid
\subsubsection{Content Based Filtering}
Das CBF basiert auf der Umwandlung von Text in mathematische Vektoren.
\subsubsection{Collaborative Filtering}

\subsubsection{Hybrid Filtering}

% Cold Start, Vielfalt, Skalierbarkeit
\subsubsection{Das Cold Start Problem}

\subsubsection{Vielfalt}

\subsubsection{Skalierbarkeit}

\subsection{Qualitätsdimensionen und Erfolgsmetriken}
% Precision@k, Recall, NDCG, APS
% Serendipity, Diversität

\subsection{Multikriterielle Systeme und Trade-offs}
% Relevanz vs. Vielfalt
% Pareto-Front (nur Konzepte)

% Fußnote: Optional GPT-Prompt-Zitation bei Bedarf
% \footnote{Einige Abschnitte wurden durch KI-gestützte Schreibunterstützung (ChatGPT) vorformuliert und redaktionell überarbeitet.}
